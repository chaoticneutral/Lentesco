\chapter*{Layer 0: The Universe}

\section*{The World and the Physics Thereof}

\R Two separate universes. The Earth that is--no magic, because the universe is
too static

\A Magic difficult. But not impossible  

\R agreed.  

\R the other place--magic, because universal laws are more sort of guidelines
than actual rules. \\ \textbf{Geography} -- the two worlds used to be linked somehow,
now the link is very weak and inconsistent. People exist on both sides of the
link, it split apart (when the universes started to shift?) and then they grew
rapidly apart.  The Earth exists in the Middle Ages-type of era, the Magical
world is...somewhat comparable amount of development but we have to pretend
there's magic to figure out what that looks like. People in the magical world
have not spread out of the obviously habitable portion of their universe, which
is why they have no idea that on the other side of the wasteland is another
habitable part of the world. \\  

\R The magic happens, somehow. we'll get to that later.  

\A So i think we talked about manipulating reality somehow.  

\R I did propose the idea, but I'm not sure that I was sold on it.  If I recall
correctly, I suggested something about "humans can adjust physical constants to
some extent in order to do things like change the way gravity works and cause
very small-scale fission to create fireballs" or something like that, just as a
part of spitballing ideas.  That has the downside of being veeery human
specific--i don't really trust chipmunks to effectively adjust the laws of
physics.  

\A Okay, this needs to be discussed in detail. \\  

\R
Each species has magic of some sort \\
Ours is intelligence 
 

\A So we need to figure out how exactly this “magic” (ie intelligence ect 
\R ETC NOT ECT it's not short for ec tetera!
\A
) helps/contributes/ect to, um casting spells? manipulating reality? 
or however this magic manifests itself. 
 

\R Yes, to casting spells. If that means 'manipulating reality' then so be it.
\\  

\A I think this is the most important question right now. It will sort a lot of
other things out. \\   

\R I agree. The other problem i have with manipulating reality (being written
here for no particular reason) is it's going to make it too scientifically
grounded for my tastes. i want MAGIC. ←:D want to have telekinesis because I sat
there and thought FUCKING MAGIC and then suddenly the X-Wing was floating out of
the swamp.  maybe we just say "I use...spell points...to power my spells and I
make cool stuff happen because I can" for writing purposes.  Practically
speaking, what happened if we did something like just said "the limits are, you
can do small-scale violation of conservation of mass \&\& energy (which I guess
technically are the same thing?) and the violation is powered by the spell
points I have that magically regenerate themselves.  so I can make fireballs
because BOOM energy, I can shoot water out of my hands because BOOM mass and
energy, I can make a magic sword to fight off the evil garden gnomes because
BOOM mass,  I can levitate something to me because BOOM energy, I can cast slow
projectile because BOOM there goes your energy, etc.  i think that's the general
stance I want to take on it for now...which means animals are just going to have
to figure out how to interact with universal constants. they'll be able to
handle it, I'm sure.  Let the fireballs begin. \\  

\A This sounds pretty good! I still think we need to discuss it more though.  
\R certainly. if it didn't need more discussion, it'd be a finished product. \\
   

\R Possibly this characteristic of inanimate objects is the thing that is
carried over when you bond an inanimate limb to yourself (spells that have any
sort of duration now have a near-permanent existence. er, this might take a bit
of thought).  

\A This is definitely cool. It can be the characteristic of inanimate objects.
But, do all inanimate objects have the same characteristic?  

\R Overall, yes? I'm considering things like :  rock is a longer duration but
doesn't retain as much force, some metals are more forceful but the spells need
to be reinforced more frequently, alloys can improve upon both available force
retention and duration, so not identical but all broadly the same thing. \\
also, that'll provide a convenient case study to suggest what the effects are of
bonding -- one dominant characteristic that has significant bearing on the way
your spells work.  Perhaps everything has the same list of spells they can cast?
 

\A So spells don’t depend on your characteristic then. You characteristic,
enhances them somehow. Like, humans can use them smartly. Inanimate objects make
them last longer. ect  

\R Yes.  And I still picture humans having SOMETHING that they can do with them
beside just "we get to use our brains to make them be employed more
effectively", but I don't quite know what.  But generally, that's how it works,
which makes the "set list of spells" concept a bit more useful. \\

they just look very different across species boundaries due to the nature of
their added characteristic.  Possibly there are also racial differences (e.g.
genetic variation within a species) that affect the precise nature of the
characteristic. yeah, i think there are.   

\A Sigh, you are such a racist!  

\R re: the list of spells: i'm sorta tempted to go with this to some extent
because the mental image of a rat and a snake magicking fireballs at each other
is incredibly pleasing.  

\A =]  

\R Life is good.  Except for the bad guys.   

\E


